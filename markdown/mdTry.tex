\section{Header}

\subsection{Download}

\href{http://daringfireball.net/projects/downloads/Markdown_1.0.1.zip}{Markdown
1.0.1} (18 KB) -- 17 Dec 2004

\subsection{Introduction}

Markdown is a text-to-HTML conversion tool for web writers. Markdown
allows you to write using an easy-to-read, easy-to-write plain text
format, then convert it to structurally valid XHTML (or HTML).

Thus, ``Markdown'' is two things: (1) a plain text formatting syntax;
and (2) a software tool, written in Perl, that converts the plain text
formatting to HTML. See the \href{/projects/markdown/syntax}{Syntax}
page for details pertaining to Markdown's formatting syntax. You can try
it out, right now, using the online
\href{/projects/markdown/dingus}{Dingus}.

The overriding design goal for Markdown's formatting syntax is to make
it as readable as possible. The idea is that a Markdown-formatted
document should be publishable as-is, as plain text, without looking
like it's been marked up with tags or formatting instructions. While
Markdown's syntax has been influenced by several existing text-to-HTML
filters, the single biggest source of inspiration for Markdown's syntax
is the format of plain text email.

bullet list:

Foo

Foo Bar

fluff

what:

Foo

Foo Bar

fluff
