\section{Abstract}\label{abstract}

Effective management of exploited fish populations, requires accurate
estimates of commercial fisheries catches to inform monitoring and
assessment efforts. In California, the high degree of heterogeneity in
the species composition of many groundfish fisheries, particularly those
targeting rockfish (genus Sebastes), leads to challenges in sampling all
potential strata, or species, adequately. Limited resources and
increasingly complex stratification of the sampling system inevitably
leads to gaps in sample data. In the presence of sampling gaps, ad-hoc
species composition point estimation is currently obtained according to
historically derived ``data borrowing'' (imputation) protocols which do
not allow for uncertainty estimation or forecasting. In order to move
from the current ad-hoc ``data-borrowing'' point estimators, we have
constructed Bayesian hierarchical models to estimate species
compositions, complete with accurate measures of uncertainty, as well as
theoretically sound out-of-sample predictions. Furthermore, we introduce
a computational method for discovering consistent ``borrowing''
strategies across over-stratified data. Our modeling approach, along
with a computationally robust system of inference and model exploration,
allows us to start to understand the effect of the highly stratified,
and sparse, sampling system on the kinds of inference possible, while
simultaneously making the most from the available data.

\section{Significance}\label{significance}

In order to understand how fish populations respond to fishing, it is
critical to obtain accurate estimates of how many fish are removed from
the ocean (catch) and to quantify the precision of those estimates.
Traditionally, population dynamics models used to measure this response
to fishing (``stock assessments'') are conditioned on a time series of
annual catches. These catch estimates are often treated as being known
without error, despite the fact that they are derived from sampling
programs that estimate the proportion of different species found within
multiple sampling strata. Sampling error introduces uncertainty into
estimates of the catch, and unsampled strata must be ``filled in''
through a process sometimes referred to on the U.S. West Coast as
``borrowing'' (i.e.~data imputation). Historically, methods used to
``borrow'' information among strata have been ad-hoc in nature and
driven by expert opinion of local managers (Sen et al. 1984, 1986)
(Pearson and Erwin 1997). We seek to improve upon this practice through
development of a model-based approach that provides estimates of catch
and associated uncertainty, as well as an objective, defensible
framework for model selection and data imputation. Although the
theoretical basis for a model based estimation of species composition in
mixed stock fisheries has been advanced (Shelton et al., 2012), it has
not yet been implemented successfully using actual historical or
contemporary data.

The difficulties associated with the existing ad-hoc approach are
magnified by an increase in the number of sampling strata over time,
specifically the number of ``market categories,'' into which fishermen
and dealers sort their catch (Figure 1, Bottom). The increase in the
number of market categories (sampling strata) has not been matched by
increases in sampling effort, resulting in a decline in the average
number of samples per stratum (Figure 1, Middle). In other words, data
are becoming more sparse, increasing our uncertainty in estimates of
catch. Since the data are also stratified over a number of ports,
fishing gear types, years, and quarters, inference is not possible
without some sort of stratum pooling. Rather than rely so heavily on the
previous, ad-hoc pooling rules which change based on the availability of
samples, we hope to standardize any necessary pooling through an
exhaustive search of the space (possible configurations) of pooled
models. Pooling (and partial pooling) among strata is achieved using
Bayesian hierarchical statistical models and model averaging (Gelman et
al., 2014).

\section{Methods}\label{methods}

\subsection{Model}\label{model}

For a particular market category, \(y_{ijklm\eta}\) is the \(i^{th}\)
sample of the \(j^{th}\) species' weight, in the \(k^{th}\) port, caught
with the \(l^{th}\) gear, in the \(\eta^{th}\) quarter, of year \(m\).
The \(y_{ijklm\eta}\) are said to be observations from a Beta-Binomial
distribution (\(BB\)) conditional on parameters \(\theta\) and \(\rho\).

\[y_{ijklm\eta} \sim BB(y_{ijklm\eta}|\theta, \rho).\]

Given observed overdispersion relative to the Poisson and Binomial
distributions, the Beta-Binomial model makes use of a correlation
parameter, \(\rho\), to better model uncertainties. The linear predictor
parameters, \(\theta\), are then factored as follows among the many
strata,

\[\theta_{jklm\eta} = \beta_0 + \beta^{(s)}_j + \beta^{(p)}_k + \beta^{(g)}_l + \beta^{(y:q)}_{m\eta}.\]

Our priors are largely diffuse, representing relatively little prior
information, producing behavior similar to classical fixed effect models
on species (\(\beta^{(s)}_{j}\)), port (\(\beta^{(p)}_{k}\)), and gear
(\(\beta^{(g)}_{l}\)) parameters. Our priors on time parameters
(\(\beta^{(y:q)}_{m\eta}\)) are normal distributions cenetered at zero
with a hierarchical variance shared among all year-quarter interaction
terms. In recent years, inference on these models has become faster and
easier to compute through the use of computational Laplace
approximations (Rue et al., 2009); we compute inferences on the above
model in R (R Core Team, 2015) using the R-INLA package (Rue et al.,
2013).

\section{Species Composition}\label{species-composition}

Applying the Bayesian predictive framework to the above model gives the
following expressions for predicted weight in each stratum,

\[p(y^*_{jklm\eta}|y) = \int\!\!\!\!\int\! \text{BB}\Big( y^*_{jklm\eta}|\theta_{jklm\eta}, \rho \Big) P\Big(\theta_{jklm\eta}, \rho | y\Big) d\theta_{jklm\eta} d\rho.\]

\(p(y^*_{jklm\eta}|y)\) is computed via monte carlo integration and
represents the model's full predictive distribution for the \(j^{th}\)
species' weight, in the \(k^{th}\) port, caught with the \(l^{th}\)
gear, in the \(\eta^{th}\) quarter, of year \(m\). The following joint
transformation of the species' predictive weights result in predictive
species compositions,

\[\pi^*_{jklm\eta} = \frac{y^*_{jklm\eta}}{\sum_j y^*_{jklm\eta}} ~~~ y^*_{klm\eta}\neq 0.\]

Because the \(y^*\) are random variables, and \(\pi^*\) is nothing more
than a transformation of the \(y^*\), \(\pi^*\) is also a random
variable. Furthermore once inference is complete, we can easily sample
these distributions and compute any desired moments from the samples.

\section{Model Exploration \&
Averaging}\label{model-exploration-averaging}

The straight-forward spatial model implied by the typical categorical
port-complex variables do not adequately make in-sample predictions at
the observed sample sizes. Presently the vanishingly small
within-stratum sample sizes are managed by an ad-hoc ``borrowing''
protocol outlined by Pearson and Erwin (1997).

We aim to formalize the idea of ``borrowing'' via an exhaustive search
of spatially pooled models among port-complexes. We combine this
exhaustive search of the set of possible pooled models, with the
formalized process of Bayesian Model Averaging (BMA) (Hoeting et al.,
1999), to integrate model uncertainty about port groupings into species
composition estimates. BMA both captures the model uncertainty around
port pooling into species composition estimates, while pinning the
relative predictive accuracy of each model against other models, in the
set of possible pooled models, to discover optimal port super-complexes
in each market category.

\begin{thebibliography}{1}

%
\bibitem{gelman} Gelman, A., Carlin, J. B., Stern, H. S., \& Rubin, D. B. (2014). Bayesian data analysis (Vol. 2). Boca Raton, FL, USA: Chapman \& Hall/CRC.

%
\bibitem{bma} Hoeting, J. A., Madigan, D., Raftery, A. E., \& Volinsky, C. T. (1999). Bayesian model averaging: a tutorial. Statistical science, 382-401.

%
\bibitem{pearsonErwin} Pearson, D.E., and Erwin, B. (1997). Documentation of California’s commercial market sampling data entry and expansion programs. NOAA Tech Memo. NOAA-TM-NMFS-SWFSC-240.

%
\bibitem{rCoreTeam} R Core Team (2015). R: A language and environment for statistical computing. R Foundation for Statistical Computing, Vienna, Austria. URL https://www.R-project.org/.

%
\bibitem{inlaPackage} Rue H., Martino S., Lindgren F., Simpson D., Riebler A. (2013). R-INLA:
Approximate Bayesian Inference using Integrated Nested Laplace
Approximations. Trondheim, Norway. URL http://www.r-inla.org/.

%
\bibitem{inlaMethod} Rue, H., Martino, S., \& Chopin, N. (2009). Approximate Bayesian
inference for latent Gaussian models by using integrated nested Laplace
approximations. Journal of the royal statistical society: Series b
(statistical methodology), 71(2), 319-392.

%
\bibitem{senMemo} Sen, A.R. (1984). Sampling commercial rockfish landings in California. NOAA Tech Memo. NOAA-TM-NMFS-SWFSC-45.

%
\bibitem{senPaper} Sen AR. (1986). Methodological problems in sampling commercial rockfish landings. Fish Bull. 84: 409-421 .

%
\bibitem{sheltonEtAl} Shelton, A. O., Dick, E. J., Pearson, D. E., Ralston, S., \& Mangel, M. (2012). Estimating species composition and quantifying uncertainty in multispecies fisheries: hierarchical 
Bayesian models for stratified sampling protocols with missing data. Canadian Journal of Fisheries and Aquatic Sciences, 69(2), 231-246.

\end{thebibliography}
